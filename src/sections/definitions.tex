\begin{itemize}
    \item \textbf{Токен доступа (Access Token)} - Это токен, используемые для авторизации на защищенном ресурсе.
    Чаще всего токен доступа в JWT-формате, то есть в виде нескольких последовательных частей, разделенных точками.
    Каждая часть содержит base64url-закодированное значение.
    \item \textbf{Токен обновления (Refresh Token)} - Это токен, используемые для получения новых токенов доступа,
    когда текущий токен доступа становится недействительным или истекает срок его действия.
    Токены обновления выдаются клиенту сервером авторизации.
    \item \textbf{Владелец ресурса (Resource Owner)} - Субъект, способный предъявить токен доступа для получения доступа к
    защищенному ресурсу.
    Когда владельцем ресурса является человек, его называют конечный пользователь.
    \item \textbf{Сервер ресурсов (Resource Server)} - Сервер, который хранит защищенные ресурсы и может обрабатывать
    запросы к этим ресурсам только после проверки наличия соответствующего токена доступа.
    К примеру если вы реализуете Microsoft OAuth 2.0, то сервером ресурсов будет являться одна из Web API компании Microsoft.
    \item \textbf{Клиент (Client)} - Приложение, выполняющее запросы к защищенным ресурсам.
    Термин не подразумевает каких-либо конкретных характеристик реализации, клиентом может быть как сервер, так десктоп
    приложение и так далее.
    \item \textbf{Сервер авторизации (Authorization Server)} - Сервер, выдающий токен доступ клиенту после
    успешной аутентификации.
\end{itemize}