Реализация обновления токенов пользователя крайне проста.
Нам нужно создать фоновый сервис~\cite{microsoftHostedservice2023}, который будет каждые несколько минут выбирать из
базы данных те сессии, которые скоро истекут,
затем фоновый сервис должен десериализовать объект \texttt{AuthenticationTicket}~\cite{microsoftAuthenticationTicket2023}
у каждой найденой сессии, из десериализованного объекта
сервис возьмет токен обновления и отправит запрос на сервер авторизации с целью получения новых токенов доступа,
обновления и ID-токена.
Новые токены заменяют старые в объекте \texttt{AuthenticationTicket}, после чего объект нужно заного сериализовать и
установить уже сериализованный объект в свойство \texttt{Value}.
Кроме того, в ответе сервера авторизации в поле \texttt{ExpiresAt} будет число, которое говорит о том через какое время
(в секундах) истечет токен доступа,
фоновый сервис должен обновить свойство ExpiresAt у объекта \texttt{UserSessionEntity}UserSessionEntity, добавив
к текущему времени секунды, полученные из ответа запроса.

Помимо обновления пользовательских сессий, фоновый сервис отвечает за удалений сессий, которые долго не использовались.
Каждые несколько минут выбираются сессии, их свойства \texttt{DateOfLastAccess} сравнивается с текущем временем,
в случае если разница между двумя датами больше 3 суток - сессия удаляется.
Каждый раз когда пользователь совершает действие на сайте, свойство \texttt{DateOfLastAccess} обновляется.

Реализация фонового сервиса может быть выполнена в соответствии с приведенными
ссылками~\cite{backroundService_2023, configurationBackgroundService_2023}.