В данной статье мы рассмотрели проблему безопасного хранения и передачи токенов доступа между микросервисами.
Особое внимание было уделено возможным уязвимостям при передаче токенов доступа, таким как как Cross-Site Scripting (XSS) и Cross-Site Request Forgery (CSRF).

Для устранения данных уязвимостей необходимо хранить авторизационные токены в файлах cookie, с обязательными
настройками \texttt{HttpOnly} и \texttt{SameSite} так что значения \texttt{SameSite} должны быть \texttt{Lax} или \texttt{Strict},
таким образом файлы куки передаются либо на безопасные \texttt{HTTP} методы, либо не передаются вовсе.

Аутентификация должна быть реализована с использованием протокола OIDC~\cite{siriwardenaOpenid2020, sakimuraOpenid2014}, а Authorization code flow с помощью PKCE~\cite{bradley2015rfc}.
олее подробно основной принцип работы протокола OIDC описан в главе 2.

Аутентификация пользователя происходит по протоколу Open ID Connect~\cite{siriwardenaOpenid2020, sakimuraOpenid2014} через Authorization code flow with PKCE~\cite{bradley2015rfc}.
Подробнее принцип работы протокола Open ID Connect и Authorization code flow with PKCE изложен во главе 2.

Так же в работе был предложен механизм аутентификации/авторизации основанный на ASP.NET Web API бекенде и Angular фронтенд приложении
под единым доменом.
Таким образом исключается необходимость передачи авторизационных куки на ресурсы под другими доменами.
Передача токена доступа на микросервисы происходит по средствам Reverse Proxy YARP~\cite{microsoftYarp2021}, таким образом что
токен доступа автоматически подставляется в заголовок запроса.

Кроме того, мы предложили механизм обновления токена доступа через классы \texttt{TicketStore}~\cite{microsoftIticketstore2023} и \texttt{HostedService}~\cite{microsoftHostedservice2023}.
Таким образом, \texttt{TicketStore} проверяет каждый запрос на истечение срока действия токена доступа.
В случае истечения срока действия токена доступа он обновляется с помощью микросервиса авторизации.
Хранилище \texttt{TicketStore} также хранит пары токенов доступа и обновления внутри
\texttt{AuthenticationTicket} entity~\cite{microsoftAuthenticationTicket2023}.

Кроме этого, в работе был предложен механизм обновления токена доступа через реализацию \texttt{TicketStore}~\cite{microsoftIticketstore2023} и \texttt{HostedService}~\cite{microsoftHostedservice2023}.
Таким образом, \texttt{TicketStore} отвечает за проверку каждого запроса на истечение срока действия токена доступа.
В случае истечения срока действия токена доступа, токен доступа обновляется с помощью запроса к эндпоинту сервиса авторизации.
Также задача \texttt{TicketStore} состоит в сохранении токена доступа и токена обновления, которые являются частью \texttt{AuthenticationTicket}~\cite{microsoftAuthenticationTicket2023}.
Hosted Service необходим для фонового обновления истекающих токенов доступа, чтобы поддерживать сессии пользователей активными.

В статье мы решили проблему безопасного хранения токена доступа и передачи его между микросервисами,
а так же предложили решения выявленным уязвимостям вида Cross-Site Scripting (XSS) и Cross-Site Request Forgery (CSRF).